\documentclass[hyperref]{ctexart}
\usepackage[left=2.50cm, right=2.50cm, top=2.50cm, bottom=2.50cm]{geometry} %页边距
\usepackage{helvet} %字体
\usepackage{amsmath, amsfonts, amssymb} % 数学公式、符号
\usepackage[english]{babel}
\usepackage{ctex} %中文,不同设备可能报错
\usepackage{graphicx} % 图片
\usepackage{url} % 超链接
\usepackage{bm} % 加粗方程字体
\usepackage{multirow} % 用于不规则制表\multirow 命令可以在表格中排版横跨两行以上的文本。命令的格式如下:\multirow{nrows}[bigstructs]{width}[fixup]{text}
\usepackage{booktabs} % 三线表宏包
% 伪代码 伪代码部分参考
\usepackage{algorithm}
\usepackage{algorithmicx}   
\usepackage{algpseudocode} 
\renewcommand{\algorithmicrequire}{\textbf{Input:}}  % Use Input in the format of Algorithm 
\renewcommand{\algorithmicensure}{\textbf{Output:}} % Use Output in the format of Algorithm 
% 伪代码
\usepackage{fancyhdr} %设置页眉、页脚
\pagestyle{fancy} %页眉页脚
\lhead{}
\chead{}
\lfoot{}
\cfoot{}
\rfoot{}
\usepackage{hyperref} %超链接交叉引用
\hypersetup{colorlinks, bookmarks, unicode} %彩色的链接显示,带书签
\usepackage{multicol}%正文双栏
\title{\textbf{Title}}
\author{\sffamily author1$^1$, \sffamily author2$^2$, \sffamily author3$^3$}
\date{(Dated: \today)}
\begin{document}
    \maketitle
    \noindent{\bf Abstract: }This is abstract.This is abstract.This is abstract.This is abstract.This is abstract.This is abstract.This is abstract.This is abstract.This is abstract.This is abstract.This is abstract.This is abstract.This is abstract.This is abstract.This is abstract.This is abstract.This is abstract.This is abstract.\\
     
    \noindent{\bf Keywords: }Keyword1; Keyword2; Keyword3;...
    \begin{multicols}{2}
        \section{Introduction} 
        This is introduction.This is introduction.This is introduction.This is introduction.This is introduction.This is introduction.This is introduction.This is introduction.This is introduction.This is introduction.This is introduction.
        \begin{algorithm}[H]
            \caption{ Framework of ensemble learning}
            \label{alg:Framwork}
            \begin{algorithmic}[1]
                \Require
            The set of positive samples for current batch, $P_n$;
            \Ensure
            Ensemble of classifiers on the current batch, $E_n$;
            \State Extracting the set of reliable negative and/or positive samples $T_n$ from $U_n$ with help of $P_n$;
            \label{code:fram:extract}
            \State Training ensemble of classifiers $E$ on $T_n \cup P_n$, with help of data in former batches;
            \label{code:fram:trainbase}
            \State $E_n=E_{n-1}cup E$;
            \label{code:fram:add}
            \State Classifying samples in $U_n-T_n$ by $E_n$;
            \label{code:fram:classify}
            \State Deleting some weak classifiers in $E_n$ so as to keep the capacity of $E_n$;
            \label{code:fram:select} \\
            \Return $E_n$;
            \end{algorithmic}
        \end{algorithm}
     
        \subsection{title}
        This is introduction.This is introduction.This is introduction.This is introduction.This is introduction.This is introduction.
        \subsubsection{title}
        This is introduction.This is introduction.This is introduction.This is introduction.This is introduction.This is introduction.
        \section{title}
        \noindent Equations:
        \begin{equation}
            E=mc^2
        \end{equation}
        \begin{equation}
            H\psi=E\psi
        \end{equation}\\
        $\partial\partial=0$, and
        $$\iint_S \vec{F}\cdot \vec{n}d\sigma=\iiint \nabla\times\vec{F}dV$$
        \section{Conclusion}           
        This is conclusion. This is conclusion. This is conclusion. This is conclusion. This is conclusion. This is conclusion. This is conclusion. This is conclusion. This is conclusion.This is conclusion.
        \section*{Acknowledgments}
        These are acknowledgments. These are acknowledgments. These are acknowledgments. These are acknowledgments. These are acknowledgments. These are acknowledgments.
        \begin{thebibliography}{100}%此处数字为最多可添加的参考文献数量
            \bibitem{article1}This is reference.%title author journal data pages
            \bibitem{book1}This is reference.%title author publish date
        \end{thebibliography}
    \end{multicols}
\end{document}
